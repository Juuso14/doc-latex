%%%%%%%%%%% Nomenclature for this chapter%%%%%%%%%%
\nomenclature{UOP}{University of Paris 11}
\chapter{Fine-tuning with intermediate Mathematical Constants}
\label{chap:chapter_2}
%%%%%%%%%%%%%%%%%%%%%%%%%%%%%%%%%%%%
%%%%%%%%%%%%%%%%%%%%%%%%%%%%%%%%%%%%

\section{The Arithmetical Monster Prime 137}

The pertinence of our above simple polynomial relations are not admitted by the standard
community, for instance arguing that since the proton is composite, its mass cannot enter simple
relations. The same argument is presented for the theoretical dependence of the electric constant a
with other constants g and $g\prime$, or with the energy level. These are reductionist arguments, unable to explain the fine-tuning phenomena, and leading to the sterile concept of unexplained emergences.
By contrast, the holistic approach implies immergences from a comprehension of the Cosmos. What a
pity for `immergence` to be a neologism.
The Eddington's proposal for a was the whole number 137, which intrigued many physicists for a
century, but apparently nobody signaled it has a fundamental mathematical property: it appears as a
Singular Prime in the series of the maximal primes appearing in the numerator of the harmonic
series: 3,11,5,137,7,11, showing a symmetry between the 11 supergravity dimensions and the 4 of
space-time. Indeed:
$$137 = 11^{2} + 4^{2}$$
$$\frac{11}{4} = (\frac{\lambda_{CNB}}{\lambda_{CMB}})^{3}$$
Since Riemann series are tied to the prime number distribution, it is strange that mathematicians
have not point out the primes appearing in the Harmonic series, since it is the single pole. It seems
that the basic precept `all occurs in the pole` was forgotten in this case. As ancient Egyptian used
only fractions of type $\frac{1}{n}$, they were certainly aware of this particular harmonic series $s_{5} = \frac{137}{60}$.
Indeed in appears in the Ptolemaic approximation for $\pi$: $$\frac{377}{120} = 2 + s_{5}/2$$.
Recall that the electrical constant a characterizes the force $\frac{\hbar c}{al^{2}}$ between two l - distant
elementary charges, appearing central in Atomic Physics and in many fine-tuning relations [1]. It is
misleading that physicists focused on only one property, the appearance of its fifth power in the
Hydrogen hyper-fine spectra, and call its inverse the `fine-structure constant`. It is strange also that
Eddington's Theory was rejected as soon as a appeared to be different from 137. Indeed, the
following shows that 137 plays a central role in fine-tuning analysis. One may interpret 137 + 1 as
the sum of the numbers of dimensions in the Topological Axis [3], taking into account the double
point for the superstring value n = 10, and the remarkable sum:
$$\sum_{k=7}^{k=0}(2 + 4 k ) = 2^{7}$$
So $137 = 2^{7} - 1 + 3 + 7$, the Hierarchic Combination form. But this appears also as 137 = 135 + 2,
with the dimension 2 of the String patent. In particular, one obtains the value $a \approx 137.035999119$
compatible with measurement $a \approx 137.035999139(31)$ in:
$ln137/ln(a/137) \approx (2+135/d_{e})^{2}$
meaning the ratio a/137 acts as a canonical ratio.

\section{The Arithmetical Logic : Holic Principle and Topological Axis}

In the hypothesis of an Arithmetic Cosmos, the ultimate equations must be diophantine. The
simplest one is $T^{2} = L^{3}$ , where T is a time ratio and L a length one, resolving, since 2 and 3 are co-
prime, by $T^{2} = L^{3} = n^{6}$ , meaning the classical 6D space of classical mechanics. This particularizes
the usual 3D space, but attribute 2 dimensions for the Time, in conformity with an independent
study [18].
This is the degenerate arithmetic form of the spatio-temporal holographic principle, It is also the
3rd Kepler's law, but its diophantine form gives $L = n^{2}$ , the orbit law in the Hydrogen atom and in our
Gravitational Molecule model, where the visible Universe corresponds to the first orbital,
suggesting the existence of a Grandcosmos, as the Topological Axis does also, which favors the
dimension n = 30, the natural extension of the above :
$T^{2} = L^{3} = M^{5} = n^{30}$ where M is the mass ratio. Recall that the lifetime of an unstable particle depends on the 5th power of its mass. This is called the Holic Principle, concerning only the apparent world. The entire Holic
Principle, concerning also the quantum vacuum, would involve a term $F^{7}$ , and of dimension 210.

\section{The retention of Information}

The Grandcosmos holographic reduction radius $R\prime$ shows itself an overwhelming holographic
relation with the CMB Wien wavelength $l_{CMB}$ , to $0.01\%$:
$$4\pi(R\prime/l_{CMB})^{2} \approx e^{a}$$
Since the holographic technique uses coherent radiation, this seems incompatible with the CMB
thermal character. But in a totally deterministic cosmos, there is no paradox. This question is
connected with the black hole information paradigm [20]. Independently of our approach, an
argument in favor of a total retention of information was tied to a non-evolution cosmology
[21], Moreover, we have shown that formalisms of Holography and Unitary Matrix Quantum
Physics are very similar [3]. Note that $e^{a}$ is also compatible with the half volume of the proton, with
the Planck length as unit.
So, while General Relativity and Unitary Quantum physics disagree about the nature of Space-
Time, specially the non-locality phenomena, they agree for complete determinism, leading to the collapse of the
Copenhagen statistical interpretation. The hidden variables exist really: the Cosmos ! Heisenberg
relations would be only Fourier transform manifestations of Wave Mechanics.

\section{Ubiquity of $a^{a}$}

The famous Lucas-Lehmer primality test uses the series of whole numbers $N_{n+1} = N_{n}^{2}-2$,
starting from $N = 4 = u_{3} + 1/u_{3}$ , with $u_{3} = \sqrt{3} + 2$, belonging to the Diophantine generators $u_{n} = \sqrt{n} + \sqrt{(n+1)}$., whose entire powers are close to whole numbers. One shows that $N_{n} \approx u_{3}^{(2^{q})}$, and for q = 9:
$u_{3}^(2^9) \approx (2(a2 + 2sqrt{\mu}))^{64} \approx a^{a}$
defining a to 39 ppm, where μ is the mass ratio muon/electron and the main term $2a^{2} = m_{e} c^{2}/E_{Ryd}$ is
tied to the Rydbergh energy's principal value $E_{Ryd}$ whose ratio with the Planck energy is closely
related to the Monster group cardinal order, to 1.5 ppm:
$$O_{M} e^{-1/2a} \approx (E_{P} /E_{Ryd})^{2} = \frac{\hbar Gc^{5}}{E_{Ryd}^2}$$
Indeed, by respect to the Chronon $\lambdabar_M /c \approx t_{\Theta} = 1.1333 \times 10{-104} s$, the number ot quantum events in
the above Supercycle period $T/ t_{\Theta}$ shows, with $\delta = R\prime/R$, to 0.4\%, 2\% and 0.6\%:
$$\delta \times T/t_{\Theta} \approx (e^{e})^{137} \approx O_{M}^{3} \approx 496^{60}$$
implying a liaison between $O_{M}$ , 137 and the famous String dimension 496, tied to the Higgs Boson
(see Fig.1). Note the following combination of the three dimensionless parameters, the electrical
one a, the Fermi ratio $F =\sqrt{a_{w}}$ and the Bizouard strong ratio $f \approx 8.4345$:
$$F/af \approx 496$$
Also, to 8 ppm: $lnO_{M} /2lnlnlnlnO_{M} \approx 137$, and the product of the 20 groups of the happy family tied
to the Monster shows, to 0.015\%:
$\Pi_{happy} \approx \delta \times a^{a}$. Also, with the Pell-Fermat generator $u_{1} = 1 + \sqrt{2}$:
$a^{a} \approx u 1^(3\times(2^{8}-1))$
defining a to 0.3 ppm. So the number a a establishes a connexion between $u_{1}$ and $u_{3}$ , two of the
simplest arithmetics generators. Moreover a a has been connected [3] to the canonical $e^{1/e}$ and the
5th optimal musical scale with 306 notes. This opens a new research in pure mathematics.






Conclusions: Simplicity at work

The application of the old direct scientific method, looking for fine tuning between physical
parameters leads to a return to the Perfect Cosmological Principle implying a Steady-state Cosmos,
confirmed by holographic quanto-cosmic relations. The Relativity theory is a local one and do not
apply in Cosmology: the Absolute Space-time is reestablished, realized by the Microwave Cosmic
Background, which identifies with the Grandcosmos Absolute Frame.
The standard Holographic Principle must be generalized to unities others than the Planck length,
even invoking the visible Universe wavelength in 1D holography, which breaks another taboo of
current thinking: the Planck wall, by an enormous factor, about 10 61 resolving the vacuum energy
dilemma factor 10 122 .
The multiple connexions with the DNA chain seems to imply it is a 1D hologram. This seems to
be confirmed by recent studies [30].
The simplest method of looking for simple monomial expressions involving mathematical
constants leads to ppb correlations, confirming Cosmos Unicity. As Atiyah writed [24]: 'Nobody has
ever wondered what the Universe would be if π were not equal to 3.14159.... Similarly no one
should be worried what the Universe would be if a were not 137.035999.... ' This is a definite
refutation of the Multiverse Hypothesis.
The present article confirms also the Topological Axis, which was obtained by the simplest
visualizing method to represent in a single figure the characteristic lengths in macro and micro-
physics, taking the electron wavelength as unity. The pertinence of the Topological Axis confirms
the importance of the Electron wavy propagation. This rehabilitates the String theory, including the
tachyonic bosonic version, since the canonical dimension 26 appears to characterizes the observable
universe radius R. This confirms that c is not a cosmic pertinent speed, as is clearly shown both by
logic (it is far too slow) and quantum non-locality.
Moreover, by excluding c in the simplest tool of elementary physics, prospective dimensional
analysis, this gives immediately a very good approximation of both R/2, the cosmic temperature
and the cosmic overall periodicity, which connects with the holic dimension n = 30 in the
Topological Axis, whose apparent dissymetry suggests directly the existence of a Grandcosmos.
While it is claimed that String Theory do not connect with experiment, the Cartan-Bott periodicity
appears, showing the gauge bosons, so confirming the Standard Model of Particle Physics, but with
massive gluon, which is independently seriously considered [31].
This means also that the International System must go back to only three fundamental unities,
Mass, Length and Time. The distinction between Length and Time must be emphasized, as
Poincaré, the father of 4D Relativity Theory himself, recommended. Indeed their confusion, by
writing c = 1, impeded the fact that R is a trivial length, already present in astrophysical text-books.The simplest model, the gravitational Hydrogen molecule gives R, explaining the above 2 factor
and justifying the elimination of c, as in the Bohr model. This corresponds to a Hubble constant
70.790 (km/s)/Megaparsec, consistent with the recent measurement [4]: 72(3) Megaparsec/(km/s),
which confirms the direct novea measurement, but disagree (3σ) with the standard value.
The simplest statistical theory of Eddington gave another justification to R. Also, particularly
simple and elegant is the Large Eddington number, giving correctly the number of neutrons in the
trivial fraction 3M/10 of the observable universe.
The simplest topological equations, the equality between dimensionless varieties, circumference,
area, 3D volume... appear to apply in cosmology, which is, for many, the hardiest chapter of
physics. This modern, negative, opinion is in fact contrary to the ancient culture, for which the
Cosmology is the first of all science, so must be the simplest. In the original sens of the word
'revolution', it is a return to the source of Science, the 'all is whole number', of Pythagoras. Even the
degenerate form of topological or holographic relations, the simplest diophantine equations, the
Holic Principle, shows direct pertinence. In particular it emphatizes the 30 dimensions, which
appear decisive in the Topological Axis.
The simplest proof of the computation basis character of the electrical parameter a is provided
by the multiple appearance of the terms $e^a$ and $a^a$ . The later is of order $e^{p/e}$ , while $e^{1/e}$ is decisive for
the operational definition of e. The fact that a a appears also in the fifth Optimal (305 notes) Musical
Scale indicates a liaison with Arithmetics.
Now, the deep significance of a number of dimensions is the number of independent variables,
which is a fundamental invariant, whatever the theory [32]. So, it is normal to introduce the
hypothesis that 26 physical parameters are defined by the 26 sporadic cardinal orders. Since
Sporadic Groups are associated with octonion algebra [33], this rejoins a prediction of Atiyah's last
work, the essential role of octonion algebra in the final theory [24].
The ancestral problem of the stability of the solar system must be revisited, taking into account
seriously a cosmic influence, characterized by the Kotov's period and length. Also the Pioneer, Tifft
and Arp effects must be seriously considered, and used to constrain the flickering Time-Length-
Mass process.
It is so previsible that the very large infra-red telescopes in preparation will show in the very far
field old-type galaxies instead of young ones. Then no artifice, as inflation, black energy,
multiverse, will not save the already refuted standard evolutionary model.
It is now clear that present mathematics are incomplete, and this article announces a
Reunification of Philosophy, Mathematics, Physics, Informatics and Theoretical Biology.
Acknowledgements. The authors want to salute the memory of Sir Michael Atiyah for this message
to one of us; 'While I appreciate your efforts in physics, please do not use my name in any way
other than referencing a published paper'. We salute his modesty, but his introduction of the rather
unexpected Euler-Mascheroni constant in the fine-tuning research has considerably helped our task
of proving the existence of a fundamental theory.

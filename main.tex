%&latex2e


\documentclass[twoside,draft]{article}
\usepackage{latexsym,e-journal}

\usepackage[letterpaper]{geometry}
\usepackage[utf8]{inputenc}
\usepackage[english]{babel}

\usepackage{amssymb,amsfonts,amsmath}

\usepackage[perpage,symbol*]{footmisc}
\usepackage[final]{graphicx}
\usepackage{pstricks}
\usepackage{cite}

\usepackage[varg]{txfonts}

\oddsidemargin=-0.20in
\evensidemargin=-0.20in
\topmargin=-30pt

\textwidth=498pt
\textheight=646pt


\begin{document}

\renewcommand{\refname}{References}
\renewcommand{\tablename}{\small Table}
\renewcommand{\figurename}{\small Fig.}
\renewcommand{\contentsname}{Contents}


\twocolumn[%
\begin{center}
\renewcommand{\baselinestretch}{0.93}
{\Large\bfseries BACK TO COSMOS

}\par
\renewcommand{\baselinestretch}{1.0}
\bigskip
F. M. Sanchez, \ V. Kotov, \ M. Grosmann, \ D. Weigel, \ R. Veysseyre, \ C. Bizouard, \ N. Flawisky, \ D. Gayral, \ L. Gueroult\\
\par
\medskip
{\small\parbox{11cm}{%
$\lambdabar_e$
To the memory of Sir Michael Atiyah\\
The ancestral concept of Cosmos is rediscovered through the idea of a Tachyonic Grandcosmos Multibasis Computer, inversing the Anthropic Principle and reestablishing the Laplace Determinism.
The observed fine tuning between Physical dimensionless parameters is interpreted as relations between optimal calculation basis, announcing a dramatic progress in computer software.
The three type of mathematical constants are considered: Large, Intermediate and close to unity. The famous Large Number problem is resolved by Eddington's statistical theory and the gravitational Hydrogen Molecule model, leading to the visible Universe horizon radius R $\approx$ 13.812 Giga ligth years.
The extension of the double cosmic correlation defines a Topological Axis using Euler-Napier constant e as primary basis, confirming String Theory, Cartan-Bott periodicity and the 30 holic dimensions corresponding to the simplest physical diophantine equation T$^2\!$ = L$^3\!$ = M$^5\!$ = N$^{30}\!$.
The point n = 30 corresponds to the common time, about 10$^{58}\!$ s, given by two mandatory dimensional analysis, interpreted as the Supercycle period.
The visible Universe wavelength 'Topon' 2G$\hbar$/Rc$^3\!$ $\approx$ 4 $\times$ 10$^{-96}\!$ m, corresponds to n $\approx$ 2e$^e\!$ and enters the 1D mono-radial holographic extension of the Bekenstein-Hawking Universe entropy, implying the critical condition, and breaking the Planck wall by a factor 10$^{61}\!$.
The monochrome holographic extension leads to a Grandcosmos, larger than the visible Universe by the same factor 10$^{61}\!$.
This implies a tachyonic speed in the same ratio by respect to c, justifying the Planck renormalisation of the Vacuum Energy, independently checked by the Casimir effect.
The couple Universe-Grandcosmos is confirmed by a dramatic geo-dimensional analysis, where Length, Time and Mass are considered as unit vectors in a 3D Super-space. A mater-antimater 10$^{104}\!$ Hz Oscillatory Bounce unifies standard Single Bang with steady-state cosmology, but suppress Relativity in cosmology, reestablishing the Newton Absolute Space-Time realized by the Microwave Cosmic Radiation.
This new space-time structure is tied to unexplained phenomena such as Kotov cycle, Tifft periodicity, Pioneer anomaly and Arp observations.
The Kotov period, omnipresent in astrophysics, is directly related with the single-electron cosmology by a 10$^{-9}\!$ relation.
In addition to e, the Archimedes constant $\pi$ and the electric constant a $\approx$ 137.036 appear as privileged calculation basis.
The fusion Mathematics-Physics is confirmed by 10$^{-9}\!$ precise relations with physical parameters implying Eddington's 137 and Atiyah constant.
The later enters the Topological axis in liaison with the canonical Galaxy Group radius, the graviton mass and the Higgs Boson, the later being associated to the string dimension d = 496. The dramatic connection of Atiyah constant with the Sternheimer scale factor confirms the fusion of Computation Physics with Theoretical Biology. In particular, the Supercycle Period connects, through the Monster group cardinal, with the Kotov period, itself tied to the well-defined DNA bi-codon mass. The ratio between the Universe mass and the mass associated with the Kotov period, identified with a photon mass, is close to the square of the Monster order. The ratio between the Supercycle period and the Chronon quantum is close to the third power of the Monster order, also close to e$^{137e}\!$ and d$^{60}\!$. It is predicted that the future infra-red spatial telescopes will find mature galaxies in the very far range, instead of a Dark Space, ruining definitely the standard evolutionary cosmology, ill-founded on an imperfect Cosmological Principle.
%% TODO: Insérer axe topologique
}}\smallskip
\end{center}]{%


\setcounter{section}{0}
\setcounter{equation}{0}
\setcounter{figure}{0}
\setcounter{table}{0}
\setcounter{page}{1}


\markboth{Your Full Names. Short Title of Your Paper}{\thepage}
\markright{Your Full Names. Short Title of Your Paper}

\section{Introduction: Deterministic Computation and Hierarchy Principle}
It was observed that the physical constants are tightly contrived, but only three dimensionless parameters: a, p, and aG, are sufficient to explain the main structures of the world [1]. Two of them are precisely measured: the electric constant a $\approx$ 137.035999139(31) measured with 0.23 ppb precision and the proton-electron mass ration p $\approx$ 1836.15267245(75), known with 0.41 ppb precision. By contrast, the gravitational coupling constant a$_G\!$ was neither well defined nor measured, due to the relatively large imprecision on G measurement (10$^{-4}\!$).

One can read [1]: \textit{“For example, the size of a planet is the geometric mean of the size of the Universe and the size of an atom; the mass of man is the geometric mean of the mass of a planet and the mass of a proton. Such relationships, as well as the basic dependences on a and aG from which they derive, might be regarded as coincidences if one does not appreciate that they can be deduced from known physical theory, with the exception of the Universe, which cannot be explained directly from kwown physics.... This line of arguments, which is discussed later, appeals to the 'anthropic principle'.”}

The existence of relations that are not explained by known physical theories, is called 'fine tuning' phenomena. But as soon as it involves the observable Universe radius, it signals the existence of a more fundamental theory that must take into account the \textit{ancestral Cosmos concept, which, as Eddington claimed [2], must be permanent [3]}. Extending this to the spatial homogeneity, this leads to the Perfect Cosmological Principle, the very foundation of the steady-state cosmology.

But, as about 30 dimensionless parameters appear as 'free parameters' in the Particle standard model, a large majority of theorists believe rather they are due to chance, leading to a separation between Physics and Mathematics, not to speak of Biology. Through a so-called Anthropic Principle, a majority believe in the Multiverse conundrum, a multiplicity of sterile Universes [1].

The present article shows that \textit{Physics is a part of mathematics}, refuting the Multiverse Hypothesis by precise fine-tuning between main physical and biological parameters, involving main mathematical constants in particle physics: $\pi$,  e and $\gamma$, the Euler-Mascheroni constant. Also, these relations confirm the Super-string Theory and rehabilitates the tachyonic Bosonic String Theory.

\textit{A decisive point of physics is the energy conservation.} Theorists associate it with time uniformity, but a more logical explication is that cosmos is a computer, so Intelligent Life receive a justification: to help the Cosmos computation. This Inverted Anthropic Principle answers the first of all questions : why do we ask questions ? We propose that the parameters are optimal basis in a deterministic Computing Cosmos, and they appear indeed in DNA characteristics, and three-point temperatures of Mammals and main molecules [3 and reference therein].

\textit{This reestablishes the Laplace determinism}, involving non-local hidden variables, which identify with the Cosmos, so rejecting the Copenhague statistical interpretation of quantum mechanics.

The fact that three parameters, out of about 30, are so clearly emerging means that Physics, and more generally Science, is hierarchic: \textit{one can progress in science without knowing the details of the underlying fundamental theory.}

So, when Dalton found whole numbers in chemical reactions, he was prefiguring the atoms and Chemistry. The same for Balmer, spectral lines and wave mechanics. The same for Mandeleiev, atomic masses and nuclear physics. Also, when Mandel found whole numbers in Biology, he was prefiguring Genetics. In the same manner, this article prefigures \textit{the fundamental theory which must be based on arithmetics,} indeed a characteristic of deterministic computation, which could proceed by optimal algorithms, so en lighting the above Hierarchy Principle.

This article is separated in 3 sections, corresponding to 3 classes of mathematical constants, the large, the intermediate and the small, close to unity.

The first section explains why, in the Computation Hypothesis, large numbers are necessary, so justifying at last the Cosmos vastness. In particular,  the most famous prime number of Hstory 2$^{127}\!$-1, and the Eddington's Large number = 136 $\times$ 2$^{256}\!$  are empathized. The second section will study the role of intermediate mathematical constants such as the Eddington's constant 137. The third Section involves mathematical constants, such as $\pi$ and $\gamma$.  By contrast, the optimal computation basis e is used all along, in particular as the primary basis of the Topological Axis Function f\{n\} = exp(2n/4), the secondary basis being 2, the simplest basis of all.


\section{Internal Fine Tuning and Canonical Large Numbers}
\markright{Your Full Names.  Short Title of Your Paper}

yuvtuytvu

\section{Citations}
\markright{Your Full Names.  Short Title of Your Paper}

A single citation is here: \cite{eddy}. Multiple citations are as follows \cite{bondi,Pez,La2}. A citation containing a comment is \cite[see p.\,5]{eddy}

%%%%%%%% the \cite{eddy} command generates citation number proceeded from
%%%%%%%% the label \bibitem{eddy} in the bibliography list


\markright{Your Full Names.  Short Title of Your Paper}
\section{Equations}
\markright{Your Full Names.  Short Title of Your Paper}

Here is a manual-numbered equation
$$
r\,= \sqrt{dx^{2} + dy^{2} + dz^{2}}.
\eqno \mbox{(1.1)}
$$

Here is an automatic-numbered equation
\begin{equation}
r\,= \sqrt{dx^{2} + dy^{2} + dz^{2}}.
\end{equation}

Here is an unnumbered equation
$$
r\,= \sqrt{dx^{2} + dy^{2} + dz^{2}}.
$$


Here is a double-line equation, typeset to the left side
$$
\begin{array}{ll}
%
\displaystyle
ds^{2}\,= L(r)dt^{2} - M(r)(dx^{2} + dy^{2} + dz^{2}) -\\[+8pt]  % 1st row
%
\displaystyle
- N(r)(xdx + ydy + zdz)^{2}, \\% 2nd row
\end{array}
$$


Here are automatic-designed brackets
\begin{equation}
\left( \frac{\mathrm{D} N^\alpha}{ds}\right),\quad
\left[ \frac{\mathrm{D} N^\alpha}{ds}\right],\quad
\left\{ \frac{\mathrm{D} N^\alpha}{ds}\right\},
\end{equation}
where you need in an ``empty'' bracket, if you feel to insert one-side brackets. For instance: $\left( \right.$.



Here are hand-designed brackets
\begin{equation}
\bigl( \frac{\mathrm{D} N^\alpha}{ds}\bigr),\quad
\Bigl( \frac{\mathrm{D} N^\alpha}{ds}\Bigr),\quad
\biggl( \frac{\mathrm{D} N^\alpha}{ds}\biggr) , 
\label{gensol}
\end{equation}
where is no need to insert an ``empty'' bracket, so you can mere type
\begin{equation}
\frac{\mathrm{D} N^\alpha}{ds} =
\Bigl\{ K^\alpha ; 0.
\end{equation}


%%%%%%%% [+8pt] is intendation following after the row
%%%%%%%% \displaystyle is normalsize in the fractions

%%%%%%%% this equation will be typeset to right, if use
%%%%%%%% {rr} argument istead {ll} in the preamble of the array

%%%%%%%% there is so many rows available as you feel

\markright{Your Full Names.  Short Title of Your Paper}
\section{Formulae in text}
\markright{Your Full Names.  Short Title of Your Paper}

Take operators in the \,{}\, brackets in the inline formulae, for compact typing: \,{=}\, gives $w \,{=}\, c^2 $. Write down \dots \ instead of ...


\markright{Your Full Names.  Short Title of Your Paper}
\section{Items}
\markright{Your Full Names.  Short Title of Your Paper}


An unnumbered item containing bullets is:
\begin{itemize}
\item The most general metric
\item The most general metric
\item The most general metric
\end{itemize}


Here is an unnumbered item:
\begin{itemize}
\item [] The most general metric
\item [] The most general metric
\item [] The most general metric
\end{itemize}


An Arabic-numbered item:
\begin{enumerate}
\item The most general metric
\item The most general metric
\item The most general metric
\end{enumerate}


A your-style numbered item:
\begin{itemize}
\item [A1] The most general metric
\item [A2] The most general metric
\item [A3] The most general metric
\end{itemize}


A double-level item (it is numbered, a sample):
\begin{enumerate}
\item The most general metric
  \begin{enumerate}
  \item The most general metric
  \item The most general metric
  \item The most general metric
  \end{enumerate}
\item The most general metric
\item The most general metric
\item The most general metric
\end{enumerate}


\markright{Your Full Names.  Short Title of Your Paper}
\section{References to text pages}
\markright{Your Full Names.  Short Title of Your Paper}


If you like to refer a numbered formula in the {equation} environment, input \label{nickname-of-the-formula} into the formula, so you will need to type (\ref{nickname-of-the-formula}) in the text instead of (12), for instance. Such reference will automatically be changed keeping the real number of the reference, if you reorder/remove/add formulae.

It works in only the {equation} environment --- auto numbered formulae.



\markright{Your Full Names.  Short Title of Your Paper}
\section{Cross-references}
\markright{Your Full Names.  Short Title of Your Paper}


Insert \label{myidea} in your text, then you have that page number where your label \pageref{myidea} appeared. For instance:

The general equation, see formula (\ref{gensol}) in page~\pageref{gensol}, is very good.

Don't use two or more same labels in the same document!



\markright{Your Full Names.  Short Title of Your Paper}
\section{Brackets, dividing paragraphs, etc.}
\markright{Your Full Names.  Short Title of Your Paper}


The commands `` and '' produce open-closed brackets: ``notation''.

Instead of \par one uses empty space(s) between paragraphs, because it is more visible.

Any sequence following a formula starts new paragraph.

If a paragraph ends by a formula, the next paragraph starts from the first line indented.

Text and space in formulae:
$$
\mbox{here is a text in this formula}\quad
\mbox{small space}\qquad \mbox{big space}
$$


\markright{Your Full Names.  Short Title of Your Paper}
\section{Spaces and dashes}
\markright{Your Full Names.  Short Title of Your Paper}


Einstein-Infeld, space-like, Bohr-like include single dash.

Page numbers 3--27 include double dash.

Thin spaces in text: v.\,13, no.\,24.

American long dash is---like this case.

British long dash is --- like this one.

We assumed the British case in our Journal.



\markright{Your Full Names.  Short Title of Your Paper}
\section{Normal size inside fractions}
\markright{Your Full Names.  Short Title of Your Paper}


Use ``displaystyle'' command before every line:
\begin{equation}
\begin{array}{ll}
\displaystyle
R_{p}(r) = \sqrt{\sqrt{C(r)}\left(\sqrt{C(r)} - \alpha\right)} + \\[+12pt]
\displaystyle
+ \;\alpha\ln\left|\frac{\sqrt{\sqrt{C(r)}} + \sqrt{\sqrt{C(r)} - 
\alpha}}{\sqrt{\alpha}}\right|.
\end{array}
\end{equation}

Compare it with follows
\begin{equation}
\begin{array}{ll}
%  \displaystyle
R_{p}(r) = \sqrt{\sqrt{C(r)}\left(\sqrt{C(r)} - \alpha\right)} + \\[+12pt]
%  \displaystyle
+ \;\alpha\ln\left|\frac{\sqrt{\sqrt{C(r)}} + \sqrt{\sqrt{C(r)} - 
\alpha}}{\sqrt{\alpha}}\right|.
\end{array}
\end{equation}




\section*{Acknowledgements}

Here are your acknowledgements.
%
\begin{flushright}\footnotesize
Submitted on Month Day, Year / Accepted on Month Day, Year
\end{flushright}


\begin{thebibliography}{99}\footnotesize

\bibitem{eddy} Eddington A.\,S. The mathematical
theory of relativity. Cambridge University Press,
Cambridge, 1924. % Here is referred book

\bibitem{bondi}  Bondi~H. Negative mass in General 
Relativity. \textit{Review of Modern Physics}, 1957, 
v.\,29\,(3), 423--428. % Here is referred article

\bibitem{Pez} Pezzaglia W. Physical applications of 
generalized Clifford Calculus: Papatetrou equations 
and metamorphic curvature. arXiv: gr-qc/9710027. 
% Here is referred electronic publication

\bibitem{La2}  Lambiase G., Papini G.,  Scarpetta G. 
Maximal acceleration corrections to the Lamb shift
of one electron atoms. \textit{Nuovo Cimento}, 
v.\,B112, 1997, 1003. arXiv: hep-th/9702130.
% Here is double paper-electronic published article


\end{thebibliography}
\vspace*{-6pt}
\centerline{\rule{72pt}{0.4pt}}
}


\end{document}

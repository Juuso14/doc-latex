%%%%%%%%%%% Nomenclature for this chapter%%%%%%%%%%
\nomenclature{UOP}{University of Paris 11}
\chapter{Fine-tuning with basic mathematical constants}
\label{chap:chapter_3}
%%%%%%%%%%%%%%%%%%%%%%%%%%%%%%%%%%%%
%%%%%%%%%%%%%%%%%%%%%%%%%%%%%%%%%%%%
\section {Basic mathematical constants}

Since some dimensionless physical parameters are very precisely measured, it seems obvious to look for
relations with mathematical constants different from the optimal basis e, such as $\pi$ and $\gamma \approx
0.577215665$, the Euler-Mascheroni constant, which appears already in the above single-electron
cosmic radius and the Topological Axis.

\section {The Wyler's approach}

Armand Wyler singularized a value $a_{W}$ approaching a to 0.6 ppm and confirmed the pertinence
of the Lenz approximation which plays a central role above: $p_{0} = 6\pi^{5}$ approaching p to 18.824 ppm.
A confirmation of a symmetry between a and 137 is the following relation involving H, the
Hydrogen electron mass ratio, precise to 83 ppb:
$$a/137 \approx (6\pi^{5} H)^{1/2} /p$$
One observe that the rejection of Wyler's work, due to a non-perfect formula for the p and a values, is a new
manifestation of the general neglectance of the Hierarchical Principle.

\section {The Archimedes constant $\pi$ as a calculation basis}

The above Lenz-Wyler formula has a geometrical interpretation: $6\pi^{5}$ is the product area-
volume of a square of radius $\pi$. Now, the value f{26} of the Topological Function for the String
main dimension 26 renders, to 0.1\%, the same form $f{26} \approx 6(2\pi^{2} a^{3} )^{5}$ , where $2\pi^{2}a^{3}$ is the area of a
4-sphere of radius a. Moreover, with n/p the mass ratio Neutron/Proton, to 0/3\%, 0.02\% and 1
ppm:
$$(p/n) (R/\lambdabar_{e})^{2} \approx (f{26}/6)^{2} \approx (2\pi^{2} a^{3})^{10} \approx \pi^{155}$$
The corresponding value of π in the last expression shows the fractional series 3, 7, 16, -u, with $u \approx
2 \times 137$. This confirms the above hypothesis concerning the origin of the Cosmos vastness, namely
that π is a intermediate rational calculation basis: in this case, the rational value $\pi_{0} = (355u-22)/
(113u+7)$ corresponds to the above G value to 10^{-8} accuracy.
Since $(R/\lambdabar_e )^{2}$ is also close to $2^{256}$ , within 1\%, this illustrates the following musical relation
involving again 137:
$$2^{1/155} \approx \pi^{1/256} \approx (2\pi)^{1/3 \times 137}$$
The scale with 155 notes is not known, but 137 appears also in the classical musical scales [3], in
particular the fifth 306 notes scale (about π 5 ), in liaison with the canonical definition of the optimal
scale e, encountered all along above, and confirmed below.
Note that entire powers of π appears in the 2 ppm Reilly formula: $a \approx 4\pi^{3} + \pi^{2} + \pi$
Recall that whole powers of π appears also in the even order Riemann series.

\section {The Euler constant e confirmed as the optimal calculation basis for the Grandcosmos}

The Topological Axis shows clearly that the Grandcosmos is defined by the following
conjonction:
$$f{e 2 } = exp(2 e2 + 1⁄2 ) \approx exp(e^{2e} + e^{2} )$$
The supplementary term exp(e 2 ) is close to a^{3/2} . Note that $e^{2}$ has the following musical property:
$$(3/2)^{5} \approx (4/3)^{7} \approx (5/4)^{9} \approx (6/5)^{11} \approx ... \approx (1+1/n)^{2n+1} \implies e^{2}$$
a series converging very much rapidly than the Euler's one $$(1+\frac{1}{n})^{n} \implies e$$. The first two terms defines
the occidental 12 tones scale.

\section {The electroweak constant mathematical fine tuning}

The Particle standard model achieved a unification between electromagnetism and weak
nuclear force. So we look for a relation involving a, 137, $a_{w}$ and the mathematical constants. One
immediately gets:
$$a_w ≈ (2γ137a/π)^3$$
Now, by introducing the characteristic length $$l_{eF} = (G_F /m_e c 2 )^{1/3}$$ , this electroweak constant appears as
a cube $a_{w} \approx (\frac{\lambdabar_{e}}{l_{eF}}^{3}$ , so:
$$\frac{\lambdabar}{l_{eF}} \approx 2γ137a/\pi$$
see below how this formulae simplifies again by using the Atiyah Constant.

\section {The Muon and Tau fine tuning}

Admitting the above relation, this defines $F = a_{w}^{1/2} = E_{F} /m_{e} c^{2} \approx 573007.3652$, inside its $2.5 10^{-7}$
indetermination. Another fine-tuning ties the muon, proton and Hydrogen masses: $\frac{E_{F}}{m_{e}\cdot c^{2}} \approx
m_{\mu}^{2} \sqrt{(m_{p} \cdot m_{H} )}/am_{e}^{3}$ . This corresponds to a muon mass relative to electron μ = 206.7682869, inside its
$2 \times 10^{-8}$ measurement range.
Now the Koide relation [22], where μ and τ are the Muon and Tau masses relative to Electron:
$$(1 + \mu + \tau)/2 = (1 + \sqrt\mu + \sqrt\tau)^{2/3}$$
has a mathematical justification in term of circulating matrix. It predicted correctly the tau/electron
mass ratio at an epoch where its measurement was false to 3 sigmas. With the above μ value, it
gives $$\tau \approx 3477.441701$$. This Koide relation, quite discarded by the communality, is another sign of
the serious incompleteness of present Particle Physics standard model. This value correlates with the
term $1+1/\sqrt{a}$, central in quantum electrodynamics (to $10^{-7}$ ):
$$1+1/\sqrt{a} \approx \tau^{3} H/pD^{2}$$
confirming the central role of the Moonshine Monster dimension [23] D = 196883.

\section {The Intermediate Bosons mathematical fine tuning}

The computer indicates, with n ≈ 1838.68366089(17) the neutron/electron mass ratio:
$$W \approx γa137 2 / 3\pi d^{e}$$
$$Z \approx ap 2 \pi^{4} / 137 d_{e}^{n}$$
With these values, the above relation $$R/\sqrt(\lambdabar_{p} \lambdabar_{H} ) \approx (WZ)^{4}$$ corresponds to the above G value in the
ppb range.

\section {The Direct Gravitational Constant mathematical fine-tuning}

Computer analysis shows the following ppb precise extension for the deviation between $2^{127}$ and
$a_{G}$ ,
$$(2^{127} / a_{G})^{1/2} \approx d_{e} (H/p)^{3} \approx a_{w}^{1/2} (a/\pi)^{4} ( \gamma/4n)^{3}$$
leading to:
$$(aa_{w}^{1/2} /\pi d e )^{1/3} \approx 4\pi n\prime/ \gamma^{a}$$
where $n\prime = nH/p$ is the principal value of the neutron mass by respect to the electron effective mass
in the Hydrogen atom. Note that this is close (0.12\%) to the monstrous 5th term 292.6345909 in
the fractional development of π which is itself very close to $n/2\pi$ to $3.4 10^{-6}$ . Since the fractional
development of π is always a non-resolved problem, this confirms that present mathematics is
incomplete and that Nature uses rational approximations for π.

\section {The Atiyah constant}

Michael Atiyah was a precursor in the quest for unicity of Mathematics and Physics. His last
work in this domain introduced the constant $$\Gamma = \gamma a/\pi$$, as a simplification term [24]. Indeed this
constant $\Gamma$ simplifies some of the above relations:
$$a_{w} = (137 \times 2 \Gamma)^3{$}$
$$W \approx 137^{2} \Gamma / 3d^{e}$$
$$( \Gamma\sqrt{a_{w}/\gamma d e )}^{1/3} \approx 4n\prime / \Gamma$$
and the above relation giving $a_{G}$ shows a dual form, the first one without any numerical factor:
$$ap_{G} / \pi \sqrt(pH) \approx (n_{F}/137^{2} \Gamma^{3} )^{3} \approx (4n/ \Gamma )^{3} /F$$
Now, as recalled before in the Holic Principle, the exponents represents the number of
dimensions. So, this corresponds to a dimensional reduction, by eliminating 137, from 9D and 6D to
3D, which could be associated to Superstring theory, where the equations are coherent only if space
has 9 dimensions, and if the 6 supplementary dimensions unfold on very small distances [25].
Bearing in mind:
$$m_{e} c^{2} f{\gamma\Gamma} \approx 125.175 GeV$$
compatible with the Higgs Boson energy, interesting to note the perfect match with the dimension index $k \approx \pi: \gamma\Gamma
\approx 4\pi + 2$. The length $\lambdabar_{e} f{\Gamma} \approx 5\times 10^{5}$ light-years is characteristic of a galaxy group radius, and the length associated to the Milankovich cycle. One obtains :
$\Gamma \approx e^\pi + 2$
meaning that the special value in the Topological Axis $k = e^{\pi} /4 \approx 4/ln2$ corresponds to about n = Γ .
The following double correlation is specially suggestive (2.2 and 0.3 ppm):
$$a \approx 4\pi^{3} + \pi^{2} + \pi \approx ln(R/\lambdabar_{e} ) + \Gamma + e^{\pi}$$
the first one is due to Reilly, and the second one confirms the R value. Moreover (0.013\%, 0.046\%):
$a - e^{\pi} \approx e^{\pi} lna \approx j = 8\pi^{2} /ln2 \approx \Theta_{mam} /\Theta_{CMB}$
where j is the Sternheimer scale factor, central in Theoretical Biology [3], being, in particular the
Temperature ratio Mammal/CMB.

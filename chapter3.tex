%%%%%%%%%%% Nomenclature for this chapter%%%%%%%%%%
\nomenclature{UOP}{University of Paris 11}
\chapter{Fine-tuning with basic mathematical constants}
\label{chap:chapter_3}
%%%%%%%%%%%%%%%%%%%%%%%%%%%%%%%%%%%%
%%%%%%%%%%%%%%%%%%%%%%%%%%%%%%%%%%%%

\chapter{Fine-tuning with basic mathematical constants}

Since some dimensionless physical parameters are very precisely measured, it is natural to look for
relations with mathematical constants different from the optimal basis e, such as π and γ ≈
0.577215665, the Euler-Mascheroni constant, which appears already in the above single-electron
cosmic radius and the Topological Axis.

\section {The Wyler's approach}

Armand Wyler singularized a value a W approaching a to 0.6 ppm and confirmed the pertinence
of the Lenz approximation which plays a central role above: p 0 = 6π 5 approaching p to 18.824 ppm.
A confirmation of a symmetry between a and 137 is the following relation involving H, the
Hydrogen electron mass ratio, precise to 83 ppb:
a/137 ≈ (6π 5 H) 1/2 /p
Note that the rejection of Wyler's work, due to a non-perfect formula for the p and a values, is a new
manifestation of the general neglectance of the Hierarchy Principle.

\section {The Archimedes constant π as a calculation basis}

The above Lenz-Wyler formula has a geometrical interpretation: 6π 5 is the product area-
volume of a square of radius π. Now, the value f{26} of the Topological Function for the String
main dimension 26 shows, to 0.1{percent}, the same form f{26} ≈ 6(2π 2 a 3 ) 5 , where 2π 2 a 3 is the area of a
4-sphere of radius a. Moreover, with n/p the mass ratio Neutron/Proton, to 0/3{percent}, 0.02{percent} and 1
ppm:
$$(p/n) (R/\lambdabar_e )^2 ≈ (f{26}/6) 2 ≈ (2π 2 a 3 ) 10 ≈ π^{155}$$
The corresponding value of π in the last expression shows the fractional series 3, 7, 16, -u, with u ≈
2×137. This confirms the above hypothesis concerning the origin of the Cosmos vastness, namely
that π is a intermediate rational calculation basis: in this case, the rational value π 0 = (355u-22)/
(113u+7) corresponds to the above G value to 10 -8 precision.
Since (R/\lambdabar_e ) 2 is also close to $2^{256}$ , within 1{percent}, this illustrates the following musical relation
involving again 137:
2 1/155 ≈ π 1/256 ≈ (2π) 1/3×137
The scale with 155 notes is not known, but 137 appears also in the classical musical scales [3], in
particular the fifth 306 notes scale (about π 5 ), in liaison with the canonical definition of the optimal
scale e, encountered all along above, and confirmed below.
Note that entire powers of π appears in the 2 ppm Reilly formula:a ≈ 4π 3 + π 2 +π
Recall that whole powers of π appears also in the even order Riemann series.

\section {The Euler constant e confirmed as the optimal calculation basis for the Grandcosmos}

The Topological Axis shows clearly that the Grandcosmos is defined by the following
conjonction:
f{e 2 } = exp(2 e2 + 1⁄2 ) ≈ exp(e 2e + e 2 )
The supplementary term exp(e 2 ) is close to a 3/2 . Note that e 2 has the following musical property:
(3/2) 5 ≈ (4/3) 7 ≈ (5/4) 9 ≈ (6/5) 11 ≈ ... ≈ (1+1/n) 2n+1
→
e 2
a series converging very much rapidly than the Euler's one (1+1/n) n → e. The first two terms defines
the occidental 12 tones scale.

\section {The electroweak constant mathematical fine tuning}

The Particle standard model achieved a unification between electromagnetism and weak
nuclear force. So we look for a relation involving a, 137, a w and the mathematical constants. One
immediately gets:
$$a_w ≈ (2γ137a/π)^3$$
Now, by introducing the characteristic length l eF = (G_F /m_e c 2 ) 1/3 , this electroweak constant appears as
a cube $a_w ≈ (\lambdabar e /l eF )^3$ , so:
\lambdabar /l eF ≈ 2γ137a/π
see below how this formulae simplifies again by using the Atiyah Constant.

\section {The Muon and Tau fine tuning}

Admitting the above relation, this defines F = a w1/2 = E F /m e c 2 ≈ 573007.3652, inside its 2.5 10 -7
indetermination. Another fine-tuning ties the muon, proton and Hydrogen masses: E F /m e c 2 ≈
m μ 2 √(m p m H )/am e3 . This corresponds to a muon mass relative to electron μ = 206.7682869, inside its
2 × 10 -8 measurement range.
Now the Koide relation [22], where μ and τ are the Muon and Tau masses relative to Electron:
(1 + μ + τ)/2 = (1 + √μ + √τ) 2 /3
has a mathematical justification in term of circulating matrix. It predicted correctly the tau/electron
mass ratio at an epoch where its measurement was false to 3 sigmas. With the above μ value, it
gives τ ≈ 3477.441701. This Koide relation, quite discarded by the communality, is another sign of
the serious incompleteness of present Particle Physics standard model. This value correlate with the
term 1+1/√a, central in quantum electrodynamics (to $10^-7$ ):
1+1/√a ≈ τ 3 H/pD 2
confirming the central role of the Moonshine Monster dimension [23] D = 196883.

\section {The Intermediate Bosons mathematical fine tuning}

The computer indicates, with n ≈ 1838.68366089(17) the neutron/electron mass ratio:
W ≈ γa137 2 /3πd e
Z ≈ ap 2 π 4 /137d e n
With these values, the above relation R/√(\lambdabar_p \lambdabar_H ) ≈ (WZ) 4 corresponds to the above G value in the
ppb range.

\section {The Direct Gravitational Constant mathematical fine-tuning}

Computer analysis shows the following ppb precise extension for the deviation between $2^127$ and
$a_G$ ,
$$(2^127 /a_G ) 1/2 ≈ d_e (H/p) 3 ≈ a w1/2 (a/ π ) 4 ( γ /4n)^3$$
leading to:
(aa w1/2 /πd e ) 1/3 ≈ 4πn'/ γ a
where n' = nH/p is the principal value of the neutron mass by respect to the electron effective mass
in the Hydrogen atom. Note that this is close (0.12 %) to the monstrous fifth term 292.6345909 in
the fractional development of π which is itself very close to n/2π to 3.4 10 -6 . Since the fractional
development of π is always a non-resolved problem, this confirms that present mathematics is
incomplete and that Nature uses rational approximations for π.

\section {The Atiyah constant}

Michael Atiyah was a precursor in the search for unity of Mathematics and Physics. His last
work in this domain introduced the constant Γ = γa/π, as a simplification term [24]. Indeed this
constant Γ simplifies some of the above relations:
a w = (137×2 Γ) 3
W ≈ 137 2 Γ / 3d e
( Γ√ a w /γd e ) 1/3 ≈ 4n'/ Γ
and the above relation giving a G shows a double form, the first one without any numerical factor:
ap G / π √(pH) ≈ (nF/137 2 Γ 3 ) 3 ≈ (4n/ Γ ) 3 /F
Now, as recalled before in the Holic Principle, the exponents represents the number of
dimensions. So, this corresponds to a dimensional reduction, by eliminating 137, from 9D and 6D to
3D, which could be associated to Superstring theory, where the equations are coherent only if space
has 9 dimensions, and if the 6 supplementary dimensions are fold on very small distances [25].
Note that:
$$m_e c 2 f{γΓ} ≈ 125.175 GeV$$
compatible with the Higgs Boson energy, Note that it corresponds to the dimension index k ≈ π: γΓ
≈ 4π + 2. The length $\lambdabar_e f{Γ} ≈ 5\times 10^5$ light-years is characteristic of a galaxy group radius, and the length associated to the Milankovich climatic period. Now :
$Γ ≈ e^π + 2$
meaning that the special value in the Topological Axis $k = e π /4 ≈ 4/ln2$ corresponds to about n = Γ .
The following double correlation is specially suggestive (2.2 and 0.3 ppm):
$$a ≈ 4π 3 + π 2 +π ≈ ln(R/\lambdabar_e ) + Γ + e π$$
the first one is due to Reilly, and the second one confirms the R value. Moreover (0.013%, 0.046%):
$a – e π ≈ e π lna ≈ j = 8π 2 /ln2 ≈ θ mam /θ CMB$
where j is the Sternheimer scale factor, central in Theoretical Biology [3], being, in particular the
Temperature ratio Mammal/CMB.
